\section{Systemsekvensdiagrammer}

I dette afsnit vil systemets sekvensdiagrammer blive beskrevet. Disse systemsekvensdiagrammer bliver fremstillet udfra user stories, som vores system er bleve indelt i.    



User stories er en måde at beskrive krav og opførsel, og er anderledes fra tasks og Use cases, i at de er beregnet til at være en simpel og effektiv måde at snakke med kunden omkring tidligere nævnt krav og opførsel. Samtidig er de med til at starte en samtale, så man kan gå fra ide til produkt, og er primært rettet mod kunden og den opførsel der skal bruges så udviklere nemmere kan se hvornår de er i mål.

Nedenfor ses sekvensdiagrammer, som er udarbejde ud fra user stories:

\begin{figure}[H]
    \centering
\includegraphics[width=0.65\textwidth]{software-architecture/sequence/Login-Register/login/login.pdf}
\caption{Viser login flowet}
\label{fig:auth}
\end{figure}

Figur \ref{fig:auth} viser login flowet, hvor brugeren har muglighed for at indtaste login information.Når brugeren har indtastet en valid login, der sendt en besked til authenicationService, hvor der bliver tjekket om passworden er korrekt eller forkert. Ud fra denne tjek kan brugeren modtage en besked på om det var en succesful login eller ikke. 

\begin{figure}[H]
    \centering
\includegraphics[width=0.70\textwidth]{software-architecture/sequence/Login-Register/signup/signup.pdf}
\caption{Viser registerings flowet}
\label{fig:user}
\end{figure}

Figur \ref{fig:user} viser registerings flowet, her vil brugeren kunne registerer sig og blive en del af Converge paltformen. Her indtastes information om brugeren og sendes til UserService, hvor der oprettes en konto og giver besked tilbage til brugeren, at kontoen er nu oprettet.  

\begin{figure}[H]
    \centering
\includegraphics[width=0.65\textwidth]{software-architecture/sequence/Employer-flow/choose-freelancer/choose-freelancer.pdf}
\caption{Viser flowet over Employer vælger en freelancer}
\label{fig:chooseAfreelancer}
\end{figure}
På \ref{fig:chooseAfreelancer} ses flowet over at en Employer vælger en freelancer efter eget ønske. Employeren trykker på knappen ''CHOOSE''  og der bliver sendt en besked til ProjectService, om at Employeren har valgt en freelancer. Herefter tjekker ProjectService om freelanceren eksisterer i systemet også bliver der sendt en besked til Employeren, om at valget er bekræftet.

\begin{figure}[H]
    \centering
\includegraphics[width=0.65\textwidth]{software-architecture/sequence/Employer-flow/delete-project/delete-project.pdf}
\caption{Viser flowet over at slette et projekt}
\label{fig:delete}
\end{figure}
Her har Employer muglighed for at kunne slette et oprettet projekt, som ses på figur \ref{fig:delete}. Her sendes en forspørgsel til ProjectService også bliver det eksisterende projekt slettet og giver herefter besked til brugeren om at projektet ikke længere eksisterer.

\begin{figure}[H]
    \centering
\includegraphics[width=0.65\textwidth]{software-architecture/sequence/Employer-flow/edit-post/edit-post.pdf}
\caption{Viser redigerings flowet}
\label{fig:edit}
\end{figure}
Figur \ref{fig:edit} viser flowet over at en employer kan fortage ændring i et eksisterende projekt. Når der fortages ændringer,
får ProjectService en besked om Employeren har ændret i projektet, de ændringer bliver gemt og der sendes en besked om ændringerne er bekræftet. 

\begin{figure}[H]
    \centering
\includegraphics[width=0.65\textwidth]{software-architecture/sequence/Employer-flow/payments/payments.pdf}
\caption{Viser betalings flowet}
\label{fig:payments}
\end{figure}
På figur \ref{fig:payments} ses flowet over at en Employer betaler for materiale via betalingssystemet Stripe. Som det aller første sendes der en anmodning til PaymentsService, derefter fortager Employeren en betaling via Stripe og til sidst modtager Employeren en besked om betaling er blevet gennemført.

\begin{figure}[H]
    \centering
\includegraphics[width=0.65\textwidth]{software-architecture/sequence/Employer-flow/post-project/post-project.pdf}
\caption{Viser flowet over at oprette et projekt}
\label{fig:projectInformation}
\end{figure}
Figur \ref{fig:projectInformation} viser hvordan flowet er, for at en Employer kan oprette et projekt. Employeren indtaster de nødvendige information, for at kunne oprette et projekt, herefter opretter ProjectService projektet og sender en besked tilbage om at projektet er oprettet. 

\begin{figure}[H]
    \centering
\includegraphics[width=0.65\textwidth]{software-architecture/sequence/Employer-flow/receive-result/receive-result.pdf}
\caption{Viser flowet over at modtage resultat}
\label{fig:receive}
\end{figure}

På figur \ref{fig:receive} ses flowet over at en Employer modtager resultater. Her anmoder Employeren om resultater, ProjectService tjekker efter om der ligger nogle resultater og hvis der gør, så sendes resultaterne til Employeren.

\begin{figure}[H]
    \centering
\includegraphics[width=0.65\textwidth]{software-architecture/sequence/Employer-flow/upload-file/upload-file.pdf}
\caption{Viser flowet over at kunne uploade filer}
\label{fig:uploade}
\end{figure}
Figur \ref{fig:uploade} viser at en Employer kan uploade en fil. Employeren ønsker uploade en file efter eget ønske, herefter behandler FileService filen og sender besked om at filen er nu uploadet og er gemt i systemet. Herefter vil der være en gennemgang af freelancer flowet, som ses på nedstående figur.

\begin{figure}[H]
    \centering
\includegraphics[width=0.65\textwidth]{software-architecture/sequence/freelancer-flow/bid-on-a-project/bid-on-a-project.pdf}
\caption{Viser flowet over at kunne byde på et projekt}
\label{fig:Bid}
\end{figure}

Figur \ref{fig:Bid} viser at en Freelancer har muglighed, for at kunne byde på et projekt efter eget præferencer. Når Freelanceren indtaster sit bud, bliver buddet sendt til BiddingsService og buddet bliver oprettet. Freelanceren vil modtage en besked om at buddet er nu oprettet, på det ønsket projekt.  


\begin{figure}[H]
    \centering
\includegraphics[width=0.65\textwidth]{software-architecture/sequence/freelancer-flow/comment-project/comment-project.pdf}
\caption{Viser flowet over at kunne kommentere på et projekt}
\label{fig:comment}
\end{figure}
På figur \ref{fig:comment} viser at en freelancer kan før en relevant dialog i forhold til et givne projekt. Her kommentere Freelanceren på et gicne projekt og kommentaren bliver sendt til CollaborationService, som gemmer buddet også giver Freelanceren besked om at kommentaren er tilføjet. 

\begin{figure}[H]
    \centering
\includegraphics[width=0.65\textwidth]{software-architecture/sequence/freelancer-flow/delete-bid/delete-bid.pdf}
\caption{Viser flowet over at kunne byde på et projekt}
\label{fig:deleteBid}
\end{figure}

Figur \ref{fig:deleteBid} viser flowet over, når freelancer ønsker og slette et eksisterende bud. Her har Freelanceren trykket på knappen ''DELETE'' og det sender en forspørgsel til BiddingsService, som tjekker om buddet eksisterer og hvis det gør skal det slettes. Freelanceren vil modtage en besked, når handlingen er gennemført. 


\begin{figure}[H]
    \centering
\includegraphics[width=0.65\textwidth]{software-architecture/sequence/freelancer-flow/employer-profile/employer-profile.pdf}
\caption{Viser Employer profil flowet}
\label{fig:profile}
\end{figure}
Figur \ref{fig:profile} viser at en Freelancer kan se Employerens profil. Freelanceren har trykket på Employer profil via ''Profil Icon'' og her sendes en anmodning til ProfilService, som finder Employerens profil og viser det for Freelanceren.

\begin{figure}[H]
    \centering
\includegraphics[width=0.65\textwidth]{software-architecture/sequence/freelancer-flow/receive-matrial/receive-matrial.pdf}
\caption{Viser flowet over at kunne modtage materiale}
\label{fig:materiale}
\end{figure}
Figur \ref{fig:materiale} viser flowet at freelancer kan modtager materiale fra en Employer. Som det første sender Freelanceren en forspørgsel til CollaborationService, hvor der tjekkes efter materiale og bekræfter at materiale er modtaget. 


\begin{figure}[H]
    \centering
\includegraphics[width=0.65\textwidth]{software-architecture/sequence/freelancer-flow/receive-payments/receive-payments.pdf}
\caption{Viser flowet over at modtage penge}
\label{fig:receivePayments}
\end{figure}
En freelancer kan modtage penge og flowet ses på figur \ref{fig:receivePayments}. Freelanceren sender en anmodning til PaymentsService om at modtage betaling, ProjectService behandler anmodning og bekræfter herefter at betaling er modtaget. 

\begin{figure}[H]
    \centering
\includegraphics[width=0.65\textwidth]{software-architecture/sequence/freelancer-flow/upload-file/upload-file.pdf}
\caption{Viser uploade fil flowet}
\label{fig:uploadeFile}
\end{figure}
Figur \ref{fig:uploadeFile}viser flowet, hvor en freelancer uploder en fil. Her uploades der en fil, som FileService behandler og uploader det, derefter bliver Freelanceren informeret om at filen er nu uploadet. 

\begin{figure}[H]
    \centering
\includegraphics[width=0.65\textwidth]{software-architecture/sequence/freelancer-flow/payments/payments.pdf}
\caption{Viser flowet over tjekke sin konto}
\label{fig:balance}
\end{figure}
På figur \ref{fig:balance} ses at en bruger har muglighed for at tjekke sin konto status. Brugeren sender som det første en anmodning til PaymentsService om at tjekke konto balancen også bliver der tjekket på kontoen og til sidst viser PaymentsService konto balancen for brugeren.
\begin{figure}[H]
    \centering
\includegraphics[width=0.65\textwidth]{software-architecture/sequence/chat/chat.pdf}
\caption{Viser chat flowet}
\label{fig:selectFriend}
\end{figure}
Figur \ref{fig:selectFriend} viser chat flowet, hvor brugeren har muglighed, for at kunne kommunikere med andre bruger. For at dette kan ske, skal brugeren først tilføje en anden bruger som kontakt og indtaste en besked. Beskeden bliver sendt til ChatService og ChatService behandler og sender beskeden til det givne bruger og til sidst vises beskeden. 

\begin{figure}[H]
    \centering
\includegraphics[width=0.65\textwidth]{software-architecture/sequence/Search/Search.pdf}
\caption{Viser search flowet}
\label{fig:search}
\end{figure}
Her kan brugeren søge efter noget specifikt og flowet ses på figur \ref{fig:search}. Her indtaster brugeren et nøgleord, der bliver behandlet af ProjectService og viser indeholdet for brugeren efter det søgte nøgleord. Til sidst kan brugeren slette det nøgle ord og indtaste et nyt nøgleord. 

\begin{figure}[H]
    \centering
\includegraphics[width=0.65\textwidth]{software-architecture/sequence/settings/settings.pdf}
\caption{Viser settings flowet}
\label{fig:settings}
\end{figure}
På figur \ref{fig:settings} ses flowet over at en brugeren kan fortage ændringer i personlige indstillinger. De ændringer der bliver fortaget sendes det til ProfilService, hvor det bliver behandlet og gemt. Brugeren modtager til sidst en besked om at ændringer er ændret.   

\begin{figure}[H]
    \centering
\includegraphics[width=0.65\textwidth]{software-architecture/sequence/cloud-native-tooling/delete-user/delete-user.pdf}
\caption{Viser slette en bruger flowet}
\label{fig:Adminflow}
\end{figure}
På figur \ref{fig:Adminflow} ses Admin flowet, hvor Admin kan slette en eksisterende bruger. Her tilgår Admin UserService og finder den korrekte bruger og sletter brugeren.

\begin{figure}[H]
    \centering
\includegraphics[width=0.65\textwidth]{software-architecture/sequence/cloud-native-tooling/restore-user/restore-user.pdf}
\caption{Viser gendanne en bruger flowet}
\label{fig:restore}
\end{figure}
Figur \ref{fig:restore} viser flowet over Admin skal gendanne en bruger. Her skal Admin som det første tilgå databasen, for at kunne finde en bestemt bruger der skal gendannes. Derefter gendannes brugeren og der kan nu ses i databasen at brugeren eksisterer i systemet igen.  

\begin{figure}[H]
    \centering
\includegraphics[width=0.65\textwidth]{software-architecture/sequence/cloud-native-tooling/user-reports/user-reports.pdf}
\caption{Viser rapport flowet}
\label{fig:rapport}
\end{figure}
Figur \ref{fig:rapport} viser flowet over, når Admin ønsker rapport over brugerens interaktioner i systemet. Her sendes en anmodning til AuditService om at der ønskes rapport på en specifikt bruger, hvor AuditService tjekker efter det givne bruger og herefter sendes rapporten til Admin.